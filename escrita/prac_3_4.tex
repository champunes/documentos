\documentclass[12pt, a4paper, oneside]{report}
\usepackage[spanish]{babel}
\usepackage[utf8x]{inputenc}
\usepackage[T1]{fontenc}
\usepackage{newtxtext}
\usepackage{etoolbox}
\usepackage[autostyle, spanish=spanish]{csquotes}
\usepackage{amsmath,amssymb,amsfonts}
\usepackage[margin=2.5cm]{geometry}
\usepackage{setspace}
\usepackage{graphicx}
\usepackage{hang}

\usepackage[table]{xcolor}

\definecolor{negro}{HTML}{000000} % #000000
\definecolor{rojo}{HTML}{ff0000} % #ff0000
\definecolor{verde}{HTML}{2da44e} % #2da44e

\graphicspath{{./img/}}

\title{Dosier de prácticas de los Temas 3 y 4}
\author{González Molina, José Ángel}
\date{\today}
\setlength{\parskip}{1em}
\setlength{\parindent}{1cm}
\onehalfspacing

\begin{document}
    \color{negro}
    \maketitle
    \clearpage

    \renewcommand{\chaptername}{Ejercicio}
    \renewcommand{\figurename}{Fig.}
    \chapter{Elabora un texto descriptivo}
        \begin{figure}[ht]
            \centering
            \includegraphics[scale = .7]{balthus}
            \caption{Balthus (1938). Thérèse soñando.}
        \end{figure}
        Al entrar en la habitación se encontró con Thérèse dormitando. La iluminación era tenue debido a que
        todas las lámparas se encontraban apagadas, por lo que poco a poco se iban vislumbrando más detalles
        de la estancia.

        Una panorámica de la escena mostraba una toalla que se encontraba descuidadamente depositada en la
        mesa. Frente a esta, una silla mal colocada. En el centro se podía observar a la pequeña recostada en
        un cojín esmeralda, el cual recogía su espalda. Finalmente en el suelo, junto a ella, un felino níveo
        como la leche; ajeno en apariencia al sopor de su compañera de juegos. Aquel día Thérèse estaba
        vestida con una camisa blanca y su habitual falda escarlata. Apoyada totalmente en el respaldo y con
        los brazos detrás de su cabeza, ojos cerrados y semblante apaciguado.

        Parecía que las tareas domésticas habían sido interrumpidas por el cansancio, o al menos eso decía el
        estado desordenado de la habitación. Pensó en despertarla, pero si no lo había hecho el ruido que
        estaba haciendo el gato con el plato, seguramente fuera porque el descanso era merecido.
        \clearpage
    \chapter{Elabora una introducción breve de un trabajo cuyo índice es el siguiente:}
        \section*{Índice}
        \begin{Large}
            \renewcommand{\labelenumii}{\arabic{enumi}.\arabic{enumii}}
            \begin{enumerate}
                \item Los orígenes del hombre
                \item Antes del \emph{Homo sapiens}. Los colonizadores del Viejo Mundo
                \item Los neandertales: ¿los primeros hombres modernos?
                \begin{enumerate}
                    \item Los primeros artistas
                    \item Pastores y agricultores
                \end{enumerate}
                \item Conclusiones
                \item Bibliografía
            \end{enumerate}
        \end{Large}
        \clearpage
        \section*{Introducción}
            En el presente trabajo se investigarán las poblaciones humanas en Europa situadas en el periodo de
            tiempo anterior al apogeo del \emph{Homo sapiens}. El \emph{Homo sapiens} es la especie a la que
            pertenecen todos los seres humanos que viven en la actualidad. Como se verá existieron multitud de
            homínidos \emph{hominina}, es decir, la familia de primates bípedos a la que pertenecemos y que contiene a
            otras especies que también poblaron Europa como el \emph{Homo antecessor}, el
            \emph{Homo heidelbergensis} y el \emph{Homo neanderthalensis}. Nos centraremos principalmente en
            los neandertales y en cómo colonizaron Europa llegando a compartir periodo temporal con nuestra
            especie.

            A lo largo de la historia de la antropología y la paleontología, ha existido una dificultad clara
            a la hora de catalogar con exactitud las especies y momentos temporales implicados en la
            colonización del continente europeo. Una de las tesis que más aceptación tiene es la capacidad de
            creación de arte o el empleo de técnicas para seleccionar animales a convertir en ganado o
            semillas que cultivar. Pero ¿son los \emph{Homo sapiens} los primeros en conseguir estos logros?

            Como se expondrá en el capítulo sobre los neandertales, las nuevas evidencias y hallazgos nos
            inducen a pensar en que dichos logros fueron logrados con anterioridad. En un primer apartado se
            describirán las últimas muestras de arte rupestre, datadas en fechas muy anteriores a los
            registros que se conservan de los primeros sapiens. Además se ha realizado un análisis de la
            complejidad y variedad de dichas obras en una comparativa a aquellas que ya conocíamos y de origen
            más reciente.

            Seguidamente, se abordará el estudio de las muestras halladas en recientes excavaciones donde se
            encontraron herramientas compatibles con un cultivo complejo de semillas. Dichas excavaciones
            están realizadas en terrenos con datación geológica muy anterior a los periodos habituales donde
            se sitúan a los humanos. También se analizarán los resultados aportados por los restos fósiles
            hallados en Irak donde se encontraron trazas de alimentos cocinados.

            Finalmente el trabajo concluye con las ideas fuerza a las que nos pueden llevar las
            investigaciones descritas con anterioridad y las consecuencias que pueden tener en nuestro
            conocimiento del origen del ser humano y cómo ha ido evolucionando hasta la especie dominante del
            mundo de la que formamos parte ahora.
        \clearpage

        \chapter{Elabora un breve estado de la cuestión a partir de la selección de textos que se presentan a continuación}
            \setcounter{section}{0}
            \section{Textos que se deben utilizar}
                \subsection*{Pérez, Antonio. \textit{El lenguaje de los homínidos}, 2009 Madrid, Cátedra.}
                    El lenguaje hablado es, hoy por hoy, el único rasgo cualitativo que diferencia
                    drásticamente a la humanidad del resto de los animales. La complejidad y funcionalidad de
                    la comunicación oral codificada entre los humanos parece estar fuera del alcance de
                    cualquier otra criatura del planeta, con excepción quizá de los cantos de las ballenas
                    corcovadas, que parecen tener connotaciones gramaticales. (p. 11)

                \subsection*{Correas, María. “Origen del lenguaje humano”, 2007. Revista \textit{Cuestiones
                científicas}, 35, pp. 1-15}
                    El lenguaje surge de forma abrupta y su emergencia está ligada a la aparición de \emph{
                    Homo sapiens}, hace 40.000 años en Eurasia. No existió un protolenguaje del que las
                    lenguas modernas evolucionaran. La aparente explosión simbólica del Paleolítico Superior
                    europeo iría acompañada de un aumento drástico en la complejidad y estructuración
                    lingüística (p. 12).

                \subsection*{Campos, Andrea. “Homínidos”, 2003. Capítulo del libro ed. por Asunción Rayo,
                \textit{Estudios de lingüística y prehistoria}. Cádiz, Universidad de Cádiz, pp. 123-155}
                    Los neandertales y demás homínidos pre sapiens poseían un lenguaje rudimentario, muy lejos
                    del lenguaje tal y como lo conocemos. Las ventajas adaptativas del lenguaje moderno serían
                    suficientes para explicar los cambios conductuales que supuestamente acaecieron durante la
                    transición Paleolítico Medio-Superior (pp. 124-125).

                \subsection*{Gutiérrez, Salvador. “Encefalización y evolución cultural”, 2004. Revista
                \textit{Lengua y Evolución}, número 7, pp. 24-67}
                    El lenguaje evoluciona gradualmente a partir de las primeras poblaciones de
                    \emph{Homo ergaster} o \emph{erectus} que comenzaron a salir de África. El estudio de
                    cráneos y réplicas internas del cerebro de \emph{Homo habilis} o de
                    \emph{Homo erectus} de hace 1,55 millones de años ha permitido afirmar que estas
                    estructuras apenas han cambiado desde entonces y que, por lo tanto, los primeros
                    \emph{Homo} gozaban de la capacidad neurológica necesaria para producir y emplear un
                    lenguaje articulado (p. 50).

                \subsection*{Sisti, Ana María. \textit{Lenguaje humano: cuándo y por qué}, 2009. Roma, Lípari}
                    En la línea continuista existe un modelo derivado de la morfología y etología comparada de
                    primates que sugiere que la necesidad de grupos mayores entre nuestros primeros ancestros
                    fue la que condujo a la evolución del lenguaje y a la encefalización de los homínidos. La
                    estrecha relación entre el tamaño de los grupos y la encefalización permite situar el
                    punto en la evolución de los homínidos en el cual el tamaño del grupo se hizo lo
                    suficientemente grande como para que el lenguaje se hiciese necesario a la hora de
                    mantener la cohesión social. Una conclusión preliminar es que el lenguaje surgió como una
                    forma de lazo de conexión más efectivo en el empleo del tiempo social. El tamaño del grupo
                    está limitado por el número de relaciones que un individuo puede controlar exitosamente,
                    número que a su vez está limitado por el tamaño relativo del neocórtex (pp. 114-15).
                    \clearpage

        \section{Estado de la cuestión}
            Existe diversa producción científica que aborda el hecho del inicio del uso del lenguaje por
            parte de los homínidos. Haciendo una revisión en profundidad se puede inferir que hay dos grandes
            posturas a la hora de ofrecer una hipótesis; a saber, que el lenguaje fue una capacidad obtenida
            por el \emph{Homo sapiens} y que sus ventajas ``serían suficientes para explicar los cambios
            conductuales'' de los que tenemos conocimiento o que, sin embargo, forma parte de un proceso
            evolutivo procedente de las especies que nos precedieron (Campos, 2003, pp.124-125).

            En defensa de la hipótesis de la aparición abrupta se encuentra el trabajo que Correas (2007) hizo
            acerca del origen del lenguaje humano donde afirma que no existió un protolenguaje anterior del
            que las lenguas que conocemos actualmente fueran evolucionando (p. 12). Además argumenta que
            el incremento del simbolismo en el Paleolítico Superior estaba acompañado de un aumento drástico
            de la complejidad lingüística (Correas, 2007, p.12). Hay otros autores que consideran al lenguaje
            algo propiamente humano como Pérez (2009), que en su libro El lenguaje de los homínidos indica que
            (el lenguaje) es el ``único rasgo que diferencia drásticamente a la humanidad del resto de
            animales'' (p. 11).

            En contraposición a lo anterior, Gutiérrez (2004) concluyó, gracias a su estudio de cráneos y
            réplicas del cerebro de especies anteriores al \emph{Homo sapiens} como el \emph{Homo erectus},
            que el lenguaje surge a lo largo de millones de años de evolución en las poblaciones de homínidos
            durante el proceso de migración desde África (p. 50). Empero, todo proceso evolutivo debe tener
            una causa o un factor de presión en el medio ambiente que lo fomente, poniendo en funcionamiento
            la selección natural. La clave para ello nos la da Sisti (2009) y su hipótesis acerca de que la
            necesidad de manejar unas relaciones más complejas en lo social, surgida conforme crecían en
            tamaño los grupos poblacionales de nuestros primates ancestrales (pp. 114-115). En su estudio
            detalla que existe una estrecha relación entre este tamaño poblacional y la encefalización,
            lo que nos lleva a pensar que el lenguaje se convirtió en una necesidad para poder mantener grupos
            tan numerosos. ``El lenguaje surgió como una forma de lazo de conexión más efectivo en el empleo
            del tiempo social'' (Sisti, 2009, pp. 114-115).

            Una vez analizadas ambas posturas surge una nueva frontera que puede desequilibrar la balanza
            hacia una u otra. Es la misma investigación de Pérez (2009) la que nos señala el foco donde poner
            la mirada. Qué complejidad lingüística debemos considerar suficiente para hablar de lenguaje
            humano propiamente dicho, ya que la comunicación entre otros animales, como las ballenas
            corcovadas, pueden tener características que nos recuerden a nuestro lenguaje (p. 11). Aunque
            Sisti nos da un mecanismo de presión del medio para forzar a los homínidos a desarrollar el
            lenguaje no significa que la respuesta evolutiva de las especies anteriores al Sapiens fueran lo
            suficientemente satisfactorias para considerarlas al mismo nivel que el lenguaje moderno.
        \clearpage

    \chapter{Prácticas de citación}
        \section{Señala y corrige los errores de uso del sistema APA}
            \begin{Large}
                \renewcommand{\labelenumi}{\alph{enumi})}
                \begin{enumerate}
                    \item Según señala Gómez (1988: pp. 24-26), la biodiversidad de las zonas tropicales está
                    en claro retroceso.

                    \textit{Es una cita indirecta narrativa, por lo que el año debe aparecer junto al autor y
                    las páginas al final de la cita, siguiendo el formato APA adecuado:}

                    Según señala Gómez (1988), la biodiversidad de las zonas tropicales está en claro
                    retroceso (pp. 24-26).
                    \clearpage

                    \item Si comparamos los resultados de las distintas investigaciones, llegaremos a la
                    conclusión de que las zonas tropicales, frente a otros ecosistemas, están en claro
                    retroceso (Gómez 1988).

                    \textit{Es una cita indirecta parentética, por lo que la cita debe aparecer al final de la
                    frase parafraseada, en el formato adecuado}

                    Si comparamos los resultados de las distintas investigaciones, llegaremos a la conclusión
                    de que las zonas tropicales, frente a otros ecosistemas, están en claro retroceso (Gómez,
                    1988).

                    \item Aparicio (2013) (Linguistics, n.º 12, p. 124) señala que ``las zonas tropicales,
                    frente a otros ecosistemas, están en claro retroceso''.

                    \textit{Esta cita es directa y narrativa por lo que la norma es mencionar el año junto al
                    autor y la página con el formato adecuado al final de la cita. El resto de información
                    aparecerá en la entrada bibliográfica.}

                    Aparicio (2013) señala que ``las zonas tropicales, frente a otros ecosistemas, están en
                    claro retroceso'' (p. 124).
                    \clearpage
                    \item Si comparamos los resultados de las distintas investigaciones, llegaremos a la
                    conclusión de que las zonas tropicales, frente a otros ecosistemas, están en claro retroceso.
                    (Gómez, 1988:3, Alvarado y Suñer, 1999:29, Carrasco, 2000:11)

                    \textit{Esta cita es indirecta y parentética, pero de varios autores que coinciden en la idea.
                    Se deben citar los autores al final de la frase en orden alfabético y con el formato
                    correcto:}

                    Si comparamos los resultados de las distintas investigaciones, llegaremos a la conclusión de
                    que las zonas tropicales, frente a otros ecosistemas, están en claro retroceso
                    (Alvarado y Suñer, 1999, p. 29; Carrasco, 2000, p. 11; Gómez, 1988, p. 3).
                \end{enumerate}
            \end{Large}
            \clearpage

        \section{Elabora una cita directa (con tres líneas y con todas las líneas) a partir de la siguiente
        información:}

            Artículo de David Gil de 2008 titulado "¿Cuán complejas son las lenguas aislantes?", de la revista
            Language, número 21, pp. 109-132.

            Las lenguas pueden ser más o menos complejas en virtud del número de elementos que posean. Se
            consideran lenguas complejas a aquellas que tienen un número alto de palabras, emplean morfemas
            para determinadas relaciones sintácticas y tienen más de diez fonemas vocálicos y veinticinco
            consonánticos. Las lenguas que tienen menos de 2.000 palabras, carecen de morfemas y poseen un
            número reducido de fonemas son lenguas simples.

            \subsection*{Texto 1}
                La complejidad de las lenguas analizar dependiendo de diversos factores. Por ejemplo, ``se
                consideran lenguas complejas a aquellas que tienen un número alto de palabras, emplean
                morfemas para determinadas relaciones sintácticas y tienen más de diez fonemas vocálicos y
                veinticinco consonánticos'' (Gil, 2008, pp. 109-132).

            \subsection*{Texto 2}
                Un autor muy instructivo a la hora de abordar la complejidad de las lenguas es Gil (2008),
                según el cual:

                \begin{hangingpar}
                    \small
                    Las lenguas pueden ser más o menos complejas en virtud del número de elementos que posean.
                    Se consideran lenguas complejas a aquellas que tienen un número alto de palabras, emplean
                    morfemas para determinadas relaciones sintácticas y tienen más de diez fonemas vocálicos y
                    veinticinco consonánticos. Las lenguas que tienen menos de 2.000 palabras, carecen de
                    morfemas y poseen un número reducido de fonemas son lenguas simples. (pp. 109-132)
                \end{hangingpar}
                \clearpage

        \section{Elabora una cita indirecta (una parentética y otra narrativa) a partir de la siguiente
        información:}

            Libro de Johanna Nichols titulado Diversidad lingüística en el espacio y en el tiempo,
            publicado en 1992 en la editorial University of Chicago Press (en Chicago).

            Existen ciertas áreas del planeta que condensan un número considerable de lenguas complejas.
            Parece, pues, que la complejidad lingüística se circunscribe a ciertas regiones. Esto significa
            que la complejidad de las lenguas puede estar relacionada con una geografía determinada, como las
            zonas aisladas (p. ej. montañas, selvas tropicales, etc.).

            \subsection*{Texto}
                En ciertos lugares del mundo se concentran numerosas lenguas, con suficiente complejidad, en
                áreas relativamente pequeñas (Nichols, 1992). Parece plausible que la geografía de un lugar
                influencie la complejidad de las lenguas que puedan surgir en dicha zona tal y como Nichols
                (1992) afirma, más aún en las zonas aisladas.

        \section{Bibliografía}
            \begin{hangingpar}
                Gil, D. (2008). ¿Cuán complejas son las lenguas aislantes?. \emph{Language}, (21), 109-132.
            \end{hangingpar}

            \begin{hangingpar}
                Nichols, J. (1992). \emph{Diversidad lingüística en el espacio y en el tiempo.} University of
                Chicago.
            \end{hangingpar}

\end{document}
