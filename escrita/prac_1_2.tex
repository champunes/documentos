\documentclass[12pt, a4paper, oneside]{report}
\usepackage[spanish]{babel}
\usepackage[utf8x]{inputenc}
\usepackage[T1]{fontenc}
\usepackage{newtxtext}
\usepackage{etoolbox}
\usepackage[autostyle, spanish=spanish]{csquotes}
\usepackage{amsmath,amssymb,amsfonts}
\usepackage[margin=2.5cm]{geometry}
\usepackage{setspace}

\usepackage[table]{xcolor}

\definecolor{negro}{HTML}{000000} % #000000
\definecolor{rojo}{HTML}{ff0000} % #ff0000
\definecolor{verde}{HTML}{2da44e} % #2da44e


\title{Dosier de prácticas de los Temas 1 y 2}
\author{José Ángel González Molina}
\date{\today}
\setlength{\parskip}{1em}
\setlength{\parindent}{1cm}
\onehalfspacing

\begin{document}
    \color{negro}
    \maketitle
    \clearpage

    \chapter{Análisis de errores en textos}
    \section{Análisis de errores}
    En este trabajo vamos a hablar % vamos a tratar / trabajar
     sobre el tema del adverbio. Haremos un resumen % realizaremos un resumen
de la lectura sobre este tema realizada por
Ofelia Kovacci. Primero hablaremos de la % Primero, trataremos la categoría
categoría y forma de los adverbios.
Seguidamente %
hablaremos % trabajaremos / trataremos
de
las
relaciones entre el adverbio y otras clases
de
palabras
como
el
adjetivo,
la
preposición y el sustantivo. Finalmente % Finalmente,
desarrollaremos las distintas clases de
adverbios de acuerdo con los criterios que
ofrece la autora.

\section{Análisis de errores}

En mi trabajo, voy a hablar acerca % En este trabajo abordaremos/trataremos
  de las distintas gramáticas que más se han extendido en España % cuidado con las generalizaciones. de la gramáticas más releveantes. de la gramática actual
  a la hora de estudiar
no solo el complemento indirecto, si no % sino
cualquier tipo de elemento que ataña a la gramática.
Para ello, voy a realizar % A lo largo del trabajo se expondrá
tres grandes bloques % exageración en grandes
donde haga un resumen de cada una de las lecturas sobre el
complemento indirecto y un apartado final % incoherente hablar de un apartado extra con los 3 anteriores
donde las ponga a comparación para poder
tener una visión contrastiva de todas las
corrientes.

\section{Análisis de errores}
El uso del lenguaje no es algo inocente % juicio de valor sin argumentación
ya que refleja tanto la diferencia sexual como la desigualdad % no se establece una definción clara de los conceptos
cultural entre hombres y mujeres.
Y aunque la gramática en sí no sea sexista % puntuación, coma
ya que el género gramatical no tiene que ver con % separar en otras oraciones
el sexo, el lenguaje, mejor dicho el uso que los hablantes hacemos de él, sí que lo es, sobre
todo porque excluye a las mujeres, dificulta su identificación o las asocia a valoraciones
peyorativas (…).
Los errores más frecuentes que corroboran el
empleo del lenguaje sexista son utilizar el
masculino tanto plural como singular para
englobar a todo el conjunto de los seres
humanos y sobre todo que aparecen muchas
El uso de la @ creo que puede estar
justificado por la difusión de las tecnologías
de la información que nos ha hecho
familiarizarnos con ella. Pero si bien puede
resultar un recurso alternativo a “os/as” en
contextos no muy formales, no veo muy
recomendable su utilización ya que no se
puede reproducir oralmente y por
supuesto no se trata de una grafía de la
lengua castellana.

\section{Análisis de errores}

En nuestra lengua, el español, % redundante
las preposiciones tiene %tienen
 un papel muy importante para realizar la función de nexo.
A lo largo de estas páginas veremos % trabajaremos, trataremos, abordaremos
 las preposiciones en español, % punto y coma
 algunos de los distintos tipos existentes, así como las
situaciones de uso más comunes y su
utilización para formar construcciones. % punto y aparte
 El
número de preposiciones en español es y
ha sido motivo de debate entre los
gramáticos,
además
de
haber
ido cambiando por épocas y modas.

\section{Análisis de errores}
El objetivo más importante de este trabajo es
aprender a diferenciar y clasificar los distintos tipos
de preposiciones, adverbios y conjunciones.
En primer lugar % coma
estudiamos el grupo de las preposiciones, incidiendo en su clasificación y su
relación con otros tipos de partículas. A través de
esto llegamos a la conclusión de que algunas
preposiciones, por ejemplo, tienen equivalentes
entre los adverbios.
A continuación, clasificaremos los adverbios
dentro de tres grupos y estudiaremos sus
características y la relación con otros elementos de
la oración.
Para terminar nos centramos en el estudio de la
conjunción: los tipos y su clasificación observando,
a la vez, la relación existente con otras partículas.

% o se unifican parrafos o se añade más información con punto y coma
% falta de coherencia en tiempos verbales

\section{Análisis de errores}
Como podemos ver, % Como se ha expuesto con anterioridad
el complemento de régimen verbal es una función todavía por % que al compartir características ...
delimitar correctamente, con muchas características
compartidas con otros complementos. El desacuerdo entre muchos
autores, de todos modos, % innecesario,
 se debe principalmente a una falta de estandarización
en la terminología. Practicamente % tilde
 cada manual usa una distinta, y a la larga eso puede
afectar a la comprensión entre los propios
estudiosos. Además, hay que admitir que la
lengua, al igual que el ser humano, y por ser
producto de estos, es algo imperfecto. Los
gramáticos deben aceptar incongruencias en
los sistemas lingüísticos puesto que provienen de seres con errores y capaces,
y que no pueden producir nada perfecto.
% no hay un orden
% no es sintético
% información nueva

\section{Análisis de errores}
(Introducción)
El contenido del trabajo trata sobre el complemento directo. % No muy bien redactado
En él se desarrolla la estructura de este complemento en las
oraciones simples y subordinadas y se expresa su relación con el
verbo.
También se habla del concepto de transitividad, muy vinculado
con el complemento directo.
Se
mencionan
las
diferentes
características
de
este
complemento, destacando el complemento directo preposicional,
muy estudiado por las distintas fuentes del trabajo ya que posee
características peculiares que son dignas de ser explicadas.
Finalmente, después de explicar las diversas características del
complemento directo y ejemplificarlas, se realiza una conclusión
al final del trabajo que trata sobre los argumentos obtenidos una
vez desarrollado y estudiado el trabajo. También hay un apartado
con las distintas diferencias expuestas por cada uno de los autores
y fuentes utilizadas en la elaboración, aquí se pueden apreciar el
enfoque que usa cada autor o los distintos términos que emplea a
la hora de estudiar el complemento directo.

% Estructuración de los parrafos

\chapter{Ejercicios de corrección y adecuación léxica}
\section{Corrige los malos usos
del léxico en el siguiente texto.}
Si bien nos iniciamos hacia un descenso % Iniciamos un descenso
demográfico que mejorará el problema de
las
plazas
universitarias
en
años
posteriores, hoy en día, de cada cien
estudiantes que empiezan COU solo
cuarenta están en disposición de hacer la
prueba de la selectividad.

\section{Corrige los malos usos
del léxico en el siguiente texto.}
La ley no trata a los drogadictos como
enfermos, sino como delincuentes, y la
sociedad los deja al amparo de la
marginación.
% restrecturar

\section{Corrige los malos usos
del léxico en los siguientes textos.}
La llegada del Presidente % presidente
 al nuevo gobierno ha suscitado % ha provocado
  la implantación de nuevos métodos con el fin de reestructurar
la economía estatal.
El gobierno anterior había impartido una % desarrollar una política
política a favor del joven % jovenes carentes
que carecía de medios económicos.

\chapter{Prácticas de ortografía}
\section{Ejercicio 1}
Rellena los huecos con
aprehensión o aprensión:
La policía llevó a cabo la --------
de los terroristas hace unos días.
Me da -------- beber del mismo
vaso.

\section{Ejercicio 2}
Escribe hojear u ojear donde
proceda:
Marta ----- los titulares del
periódico con desgana.
Conviene ----- los libros antes de
comprarlos.

\section{Ejercicio 3}
Escribe gravar o grabar donde
proceda:
Quiero ----- mi nombre en la medalla.
Me van a ----- la casa con un impuesto
muy alto.
¿Me ----- la película?

\section{Ejercicios de T. Chacón (2012), Ortografía
normativa del español. Cuaderno de
ejercicios.}

\chapter{Ejercicios de puntuación}

\end{document}
