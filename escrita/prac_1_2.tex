\documentclass[12pt, a4paper, oneside]{report}
\usepackage[spanish]{babel}
\usepackage[utf8x]{inputenc}
\usepackage[T1]{fontenc}
\usepackage{newtxtext}
\usepackage{etoolbox}
\usepackage[autostyle, spanish=spanish]{csquotes}
\usepackage{amsmath,amssymb,amsfonts}
\usepackage[margin=2.5cm]{geometry}
\usepackage{setspace}
\usepackage{ulem}

\usepackage[table]{xcolor}

\definecolor{negro}{HTML}{000000} % #000000
\definecolor{rojo}{HTML}{ff0000} % #ff0000
\definecolor{verde}{HTML}{2da44e} % #2da44e


\title{Dosier de prácticas de los Temas 1 y 2}
\author{José Ángel González Molina}
\date{\today}
\setlength{\parskip}{1em}
\setlength{\parindent}{1cm}
\onehalfspacing

\newcommand\redsout{\bgroup\markoverwith{\textcolor{red}{\rule[0.5ex]{2pt}{0.4pt}}}\ULon}

\begin{document}
    \color{negro}
    \maketitle
    \clearpage

    \chapter*{Análisis de errores en textos}
        \setcounter{chapter}{1}
        \section{Análisis de errores}
            En este trabajo \redsout{vamos a hablar} \color{verde}vamos a tratar\color{negro} \;sobre el tema
            del adverbio. \redsout{Haremos un resumen} \color{verde}Realizaremos un resumen\color{negro} \;de
            la lectura sobre este tema realizada por Ofelia Kovacci. Primero \redsout{hablaremos de}
            \color{verde}trataremos\color{negro} \;la categoría y forma de los adverbios. Seguidamente
            \redsout{hablaremos de} \color{verde}trabajaremos\color{negro} \;con las relaciones entre el
            adverbio y otras clases de palabras como el adjetivo, la preposición y el sustantivo.
            Finalmente\color{verde}\textbf{,}\color{negro} \;desarrollaremos las distintas clases de adverbios
            de acuerdo con los criterios que ofrece la autora.

        \section{Análisis de errores}
            En \redsout{mi trabajo, voy a hablar acerca de} \color{verde}este trabajo abordaremos\color{negro}
            \;las distintas gramáticas, que \redsout{más se han extendido} \color{verde}se han convertido en
            más relevantes\color{negro} \;en España, a la hora de estudiar no solo el complemento indirecto,
            \redsout{si no} \color{verde}sino\color{negro} \;cualquier tipo de elemento que ataña a la
            gramática. \redsout{Para ello, voy a realizar} \color{verde}A lo largo del trabajo se expondrán
            \color{negro} \redsout{tres} \color{verde}cuatro\color{negro} \;\redsout{grandes} bloques donde se
            haga un resumen de cada una de las lecturas sobre el complemento indirecto y un apartado final
            donde se las \redsout{ponga a} \color{verde}ponga en\color{negro} \;comparación para poder tener
            una visión contrastiva de todas las corrientes.
            \clearpage

        \section{Análisis de errores}
            El uso del lenguaje \redsout{no es algo inocente} \color{verde}parece estar influenciado por la
            cultura,\color{negro} \; ya que refleja tanto la diferencia sexual como la desigualdad cultural
            entre hombres y mujeres. \color{verde}Nos referimos en este caso a que se pueden observar en el
            lenguaje muestras de ello.\color{negro} \;\redsout{Y aunque la} \color{verde}La\color{negro}
            \;gramática en sí no \redsout{sea} \color{verde}es\color{negro} \;sexista\color{verde}\textbf{,}
            \color{negro} \redsout{ya} \color{verde}puesto\color{negro} \;que el género gramatical no tiene
            que ver con el \redsout{sexo,} \color{verde}sexo. A pesar de ello,\color{negro}
            \;\redsout{el lenguaje, mejor dicho el uso que los hablantes hacemos de él,} \color{verde}el uso
            que los hablantes hacemos del lenguaje\color{negro} \;sí que \redsout{lo es} \color{verde}puede
            ser sexista\color{negro}, sobre todo porque excluye a las mujeres, dificulta su identificación o
            las asocia a valoraciones peyorativas \redsout{(…)}.

            Los \redsout{errores} \color{verde}usos\color{negro} \;más frecuentes que corroboran el empleo del
            lenguaje sexista son utilizar el masculino\color{verde}\textbf{,}\color{negro} \;tanto plural como
            singular\color{verde}\textbf{,}\color{negro} \;para englobar a todo el conjunto de los seres
            humanos y \redsout{sobre todo que aparecen muchas} \color{verde}las abundantes\color{negro}
            \;connotaciones peyorativas \color{verde}que aparecen en el discurso. Un ejemplo de ello son las
            palabras\color{negro}\;\redsout{como} \textit{fulano}\color{verde}\textbf{,}\color{negro} \;que
            designa a una persona indeterminada \color{verde}cuando tiene género masculino,\color{negro} \;y
            \textit{fulana}, \color{verde}que designa a una\color{negro} \;prostituta \color{verde}cuando
            tiene género femenino \color{negro}\redsout{(…)}.

            El uso de la \redsout{@} \color{verde}arroba\color{negro} \;\redsout{creo que puede estar}
            \color{verde}se puede ver\color{negro} \;justificado por la difusión de las tecnologías de la
            información\color{verde}\textbf{,}\color{negro} \;\redsout{que} \color{verde}lo cual\color{negro}
            \;nos ha hecho familiarizarnos con ella. Pero si bien puede resultar un recurso alternativo a
            “os/as” en contextos no muy formales, \redsout{no veo} \color{verde}no parece\color{negro} \;muy
            recomendable su utilización\color{verde}\textbf{,}\color{negro} \;\redsout{ya} \color{verde}debido
            a\color{negro} \;que no se puede reproducir oralmente y\color{verde}\textbf{,}\color{negro} \;por
            supuesto\color{verde}\textbf{,}\color{negro} \; no se trata de una grafía de la lengua castellana.

        \section{Análisis de errores}

            En nuestra lengua\redsout{, el español,} \color{verde}castellana\color{negro} \;las preposiciones
            \redsout{tiene} \color{verde}tienen\color{negro} \;un papel muy importante para realizar la
            función de nexo. A lo largo de estas páginas \redsout{veremos} \color{verde}abordaremos
            \color{negro} \;las preposiciones \redsout{en español,} \;\color{verde}del español;\color{negro}
            \;algunos de los distintos tipos existentes, así como las situaciones de uso más comunes y su
            utilización para formar construcciones.

            El número de preposiciones en español es\color{verde}\textbf{,}\color{negro} \;y ha
            sido\color{verde}\textbf{,}\color{negro} \;motivo de debate entre los gramáticos, además de haber
            ido cambiando por épocas y modas.

        \section{Análisis de errores}

            El objetivo más importante de este trabajo es aprender a
            diferenciar\color{verde}\textbf{,}\color{negro} \;y clasificar\color{verde}\textbf{,}\color{negro}
            \;los distintos tipos de preposiciones, adverbios y conjunciones. En primer
            lugar\color{verde}\textbf{,}\color{negro} \;\redsout{estudiamos} \color{verde}trabajaremos con
            \color{negro} el grupo de las preposiciones, incidiendo en su clasificación y su relación con
            otros tipos de partículas. A través de \redsout{esto llegamos} \color{verde}dicho estudio
            llegaremos\color{negro} \;a la conclusión de que algunas preposiciones, por ejemplo, tienen
            equivalentes entre los adverbios.

            \redsout{A continuación} \color{verde}Seguidamente\color{negro}, clasificaremos los adverbios
            dentro de \redsout{tres grupos y} \color{verde}tres grupos, \color{negro} estudiaremos sus
            características y la relación con otros elementos de la oración. Para terminar
            \redsout{nos centramos} \color{verde}nos centraremos\color{negro} \;en el estudio de la
            conjunción: los tipos y su clasificación \redsout{observando}\color{verde}. A la vez
            observaremos\color{negro}\redsout{, a la vez,} la relación existente con otras partículas.

        \section{Análisis de errores}
            Como \redsout{podemos ver} \color{verde}se ha expuesto con anterioridad\color{negro},
            el complemento de régimen verbal es una función \redsout{todavía por delimitar correctamente,}
            \color{verde}que al compartir\color{negro} \;\redsout{con} muchas características
            \redsout{compartidas} con otros complementos \color{verde}dificulta su delimitación\color{negro}.
            El desacuerdo entre muchos autores\redsout{, de todos modos,} se debe principalmente a una falta
            de estandarización en la terminología. \redsout{Practicamente} \color{verde}Prácticamente
            \color{negro} cada manual usa una \redsout{distinta, y} \color{verde}distinta y,\color{negro}
            \;a la larga\color{verde}\textbf{,}\color{negro} \;eso puede afectar a la comprensión entre los
            propios estudiosos. \redsout{Además, hay que admitir} \color{verde}En consiguiente, parece sensato
            tener en cuenta\color{negro} \;que la lengua \color{verde}es un producto
            humano\color{negro}\redsout{, al igual que el ser humano, y por ser producto de estos,}
            \color{verde}e igualmente,\color{negro} \;\redsout{es} algo imperfecto. Los gramáticos deben
            aceptar incongruencias en los sistemas lingüísticos\color{verde}\textbf{,}\color{negro} \;puesto
            que provienen de seres con errores y capaces de equivocarse\redsout{, y que no pueden producir
            nada perfecto}.
            \clearpage

        \section{Análisis de errores}
            \redsout{El contenido del trabajo trata sobre el} \color{verde}En este trabajo abordaremos tanto
            la estructura del\color{negro} \;complemento directo\redsout{. En él se desarrolla la estructura
            de este complemento} en las oraciones simples y subordinadas \color{verde}como\color{negro}
            \;\redsout{y se expresa} su relación con el verbo. \redsout{También se habla del}
            \color{verde} Asimismo trataremos el\color{negro} \;concepto de transitividad, muy vinculado con
            el complemento directo. \redsout{Se mencionan} \;\color{verde}Seguidamente describiremos
            \color{negro} las diferentes características de este complemento, destacando el complemento
            directo preposicional, muy estudiado por las distintas fuentes \redsout{del trabajo}\color{verde}
            \;consultadas para la elaboración de este trabajo\textbf{,}\color{negro} \;ya que posee
            características peculiares que son dignas de ser explicadas.

            Finalmente, después de explicar las diversas características del complemento directo y
            ejemplificarlas, se \redsout{realiza} \color{verde}se llegará a\color{negro} \;una conclusión
            \redsout{al final del trabajo} que trata sobre los argumentos obtenidos \redsout{una vez
            desarrollado y estudiado el trabajo} \color{verde}gracias al análisis de los conocimientos
            recopilados\color{negro}. También hay un apartado con las distintas diferencias expuestas por cada
            uno de los autores \redsout{y fuentes utilizadas en la elaboración, aquí se pueden apreciar},
            \color{verde}donde se aprecia\color{negro} \;el enfoque que usa cada autor o los distintos
            términos que \redsout{emplea} \color{verde}emplean\color{negro} \;a la hora de estudiar el
            complemento directo.

    \chapter*{Ejercicios de corrección y adecuación léxica}
        \setcounter{chapter}{2}
        \setcounter{section}{0}

        \section{Corrige los malos usos del léxico en el siguiente texto.}
            Si bien \redsout{nos iniciamos hacia} \color{verde}iniciamos\color{negro} \;un descenso
            demográfico que mejorará el problema de las plazas universitarias en \redsout{años posteriores}
            \color{verde}los próximos años\color{negro}, hoy en día, de cada cien estudiantes que empiezan
            COU solo cuarenta están en disposición de hacer la prueba de la selectividad.

        \section{Corrige los malos usos del léxico en el siguiente texto.}
            La ley no trata a los drogadictos como enfermos, sino como delincuentes, y la sociedad los deja
            \redsout{al amparo de} \color{verde}abocados a\color{negro} \;la marginación.

        \section{Corrige los malos usos del léxico en los siguientes textos.}
            La llegada del \redsout{Presidente} \color{verde}presidente\color{negro} \; al nuevo gobierno
            \redsout{ha suscitado} \color{verde}ha provocado\color{negro} \;la implantación de nuevos métodos
            con el fin de reestructurar la economía estatal.

            El gobierno anterior había \redsout{impartido} \color{verde}desarrollado\color{negro} \;una
            política a favor \redsout{del joven que carecía} \color{verde}de los jóvenes carentes\color{negro}
            \;de medios económicos.

        \chapter*{Prácticas de ortografía}
            \setcounter{chapter}{3}
            \setcounter{section}{0}

        \section{Rellena los huecos con aprehensión o aprensión:}
            La policía llevó a cabo la \color{verde}aprehensión\color{negro} \;de los terroristas hace unos
            días.

            Me da \color{verde}aprensión\color{negro} \;beber del mismo vaso.

        \section{Escribe hojear u ojear donde proceda:}
            Marta \color{verde}hojeó\color{negro} \;los titulares del periódico con desgana.

            Conviene \color{verde}ojear\color{negro} \;los libros antes de comprarlos.

        \section{Escribe gravar o grabar donde proceda:}
            Quiero \color{verde}grabar\color{negro} \;mi nombre en la medalla.

            Me van a \color{verde}gravar\color{negro} \;la casa con un impuesto muy alto.

            ¿Me \color{verde}grabas\color{negro} \;la película?
            \clearpage

        \section{Ejercicios de T. Chacón (2012), Ortografía normativa del español. Cuaderno de ejercicios.}
            \renewcommand{\labelenumi}{\alph{enumi})}
            \subsection{Señalar la línea correctamente escrita de las siguientes:}
                \begin{enumerate}
                    \item apacigüéis, licúais, sentenciais;
                    \item apacigüeís, licuais, sentenciais;
                    \item apacigüeis, licuáis, sentenciais;
                    \item \color{verde}apacigüéis, licuáis, sentenciáis\color{negro}.
                \end{enumerate}
            \subsection{Completar, con una de las alternativas que se ofrecen, la siguiente frase mutilada:}
                Juan tiene su ... para retirarse, ... nadie le dijo... no lo avisaron.
                \begin{enumerate}
                    \item por qué / porque / por qué;
                    \item por que / porqué / porque;
                    \item \color{verde}porqué / porque / por qué\color{negro};
                    \item porque / por que / porqué.
                \end{enumerate}
            \subsection{Pon g o j donde corresponda.}
                \begin{enumerate}
                    \item co\color{verde}g\color{negro}ió, cónyu\color{verde}g\color{negro}e,
                    indí\color{verde}g\color{negro}ena, \color{verde}g\color{negro}ijonés;
                    \item su\color{verde}j\color{negro}eción, homena\color{verde}j\color{negro}e,
                    here\color{verde}j\color{negro}e, \color{verde}g\color{negro}eranio;
                    \item tra\color{verde}j\color{negro}e, extran\color{verde}j\color{negro}ero,
                    gara\color{verde}j\color{negro}e, jen\color{verde}j\color{negro}ibre;
                    \item parado\color{verde}j\color{negro}a, cru\color{verde}j\color{negro}ir,
                    condu\color{verde}j\color{negro}e, \color{verde}j\color{negro}ijonenco.
                \end{enumerate}
            \subsection{La única línea correcta integra, de las cuatro que se proponen, es:}
                \begin{enumerate}
                    \item supérfluo, élite, eburneo;
                    \item superfluo, elite, eburneo;
                    \item supérfluo, élite, ebúrneo;
                    \item \color{verde}superfluo, élite, ebúrneo\color{negro}.
                \end{enumerate}
            \subsection{Reponer en la frase las formas verbales:}
                Se... de los alimentos que no...
                \begin{enumerate}
                    \item preveyó ... comiste;
                    \item \color{verde}proveyó ... comiste\color{negro};
                    \item provió ... comiste;
                    \item proveyó ... comistes.
                \end{enumerate}
            \subsection{Indicar la línea, de las cuatro que se proponen, con escritura impecable:}
                \begin{enumerate}
                    \item antigüo, desagüe, argüyo;
                    \item antiguo, desague, argüimos;
                    \item antigüo, desagüe, paragüas;
                    \item \color{verde}antiguo, desagüe, argüía\color{negro}.
                \end{enumerate}
                \clearpage
            \subsection{La palabra raíz lleva tilde:}
                \begin{enumerate}
                    \item \color{verde}para marcar el hiato (no hay diptongo)\color{negro};
                    \item por ser aguda y acabar en consonante;
                    \item por ser aguda y bisílaba;
                    \item por ser llana y acabar en consonante.
                \end{enumerate}
            \subsection{¿Es correcta, en la frase \textit{Como como mucho, no sé cómo puedo estar delgado},
            la acentuación de la secuencia /\textit{como}/?}
                \begin{enumerate}
                    \item No, porque ninguna de las tres debería llevar acento.
                    \item No, porque debería acentuarse también el \textit{como} causal.
                    \item Sí, porque la acentuación de \textit{como} es siempre opcional.
                    \item \color{verde}Sí, porque solo el tercer \textit{como} es interrogativo\color{negro}.
                \end{enumerate}
            \subsection{¿Cuál es la puntuación apropiada en estas frases?}
                \begin{enumerate}
                    \item El torero, montera en mano\color{verde}\textbf{,}\color{negro} \;se dirigió a la
                    Presidencia.
                    \item El torero\color{verde}\textbf{,}\color{negro} \;montera en mano, se dirigió a la
                    Presidencia.
                    \item Florián, que es muy estudioso, aprobó el curso.
                    \item Florián\color{verde}\textbf{,}\color{negro} \;el del quinto, aprobó el curso.
                \end{enumerate}
                \clearpage
            \subsection{¿En qué oración de las siguientes son incorrectos los signos de puntuación o los
            auxiliares?}
                \begin{enumerate}
                    \item En Valladolid (ciudad de Castilla) murió Colón.
                    \item En Valladolid --ciudad de Castilla-- murió Colón.
                    \item En Valladolid, ciudad de Castilla, murió Colón.
                    \item \color{verde}En Valladolid, ciudad de Castilla: murió Colón\color{negro}.
                \end{enumerate}
            \subsection{¿Cuál de estas oraciones está bien puntuada?}
                \begin{enumerate}
                    \item Carlos que es mi hermano, vendrá hoy a casa.
                    \item El director, es quien manda.
                    \item Elisa, la más destacada, de las compañeras llegará hoy.
                    \item \color{verde}El Manco de Lepanto, o sea, Cervantes, escribió La
                    Galatea\color{negro}.
                \end{enumerate}
            \subsection{La frase truncada \textit{Estaban apiñados delante... y alrededor...} se completa
            con:}
                \begin{enumerate}
                    \item \color{verde}de nosotros... de ella\color{negro};
                    \item nuestro... suyo;
                    \item nuestro... de ella;
                    \item de nosotros... suya.
                \end{enumerate}
            \subsection{Resulta inapropiada la escritura de las formas (de los verbos \textit{realizar} y
            \textit{satisfacer}) siguientes:}
                \begin{enumerate}
                    \item \color{verde}realizé, satisfaciamos\color{negro};
                    \item realizarais, satisficierais;
                    \item realizó, satisfice;
                    \item realizamos, satisficieseis.
                \end{enumerate}
            \subsection{¿Faltan letras a estas palabras: ex.uberante, e.tructura?}
                \begin{enumerate}
                    \item Ninguna a la primera; una x, a la segunda.
                    \item hu y x, respectivamente.
                    \item hu y s, por este orden.
                    \item \color{verde}A la primera, ninguna; a la otra, una s\color{negro}.
                \end{enumerate}
        \section{Pon mayúsculas donde proceda.}
            \renewcommand{\labelenumi}{\arabic{enumi})}
            \begin{enumerate}
                \item Te veré este sábado.
                \item Me presenté al examen en la convocatoria de marzo.
                \item El texto aborda el estudio de la historia del \redsout{h}\color{verde}H\color{negro}omo
                sapiens.
                \item Cuando Colón llegó al nuevo mundo, creía estar en la India.
                \item El año pasado visitaron los \redsout{a}\color{verde}A\color{negro}ndes y el
                \redsout{e}\color{verde}E\color{negro}verest.
                \item Así aparece en el \redsout{r}\color{verde}R\color{negro}eal
                \redsout{d}\color{verde}D\color{negro}ecreto 125/1988.
                \item La \redsout{d}\color{verde}D\color{negro}eclaración
                \redsout{u}\color{verde}U\color{negro}niversal de los
                \redsout{d}\color{verde}D\color{negro}erechos \redsout{h}\color{verde}H\color{negro}umanos.
                \item Este año cursaré la asignatura de \redsout{l}\color{verde}L\color{negro}engua española y su didáctica.
                \item Esto ocurrió en el \redsout{p}\color{verde}P\color{negro}aleolítico.
            \end{enumerate}

    \chapter*{Ejercicios de puntuación}
    \setcounter{chapter}{4}
    \setcounter{section}{0}

        \section{Añade coma cuando sea necesario:}
        Pocas cartas me han hecho más feliz que una recién recibida con la que un conciudadano se me
        dirige\color{verde}\textbf{,}\color{negro} \;apelando a mi calidad de
        responsable\color{verde}\textbf{,}\color{negro} \;según piensa él\color{verde}\textbf{,}\color{negro}
        \;de cuantas palabras del idioma empiezan por R\color{verde}\textbf{,}\color{negro} \;pues esa es la
        letra del sillón que ocupo en la Academia. Me reclama justicia: representa\color{verde}\textbf{,}
        \color{negro}dice\color{verde}\textbf{,}\color{negro} \;a determinadas personas cuya profesión se
        designa con un nombre encabezado por dicha letra\color{verde}\textbf{,}\color{negro}
        \;restaurador\color{verde}\textbf{,}\color{negro} \;y desea verme actuar enérgicamente para impedir
        que ``su'' nombre sea empleado para designar otras actividades distintas\color{verde}\textbf{,}
        \color{negro}tal como ahora se está haciendo.
        \begin{flushright}
            (Fernando Lázaro Carreter, \textit{El dardo en la palabra}, 1997, pg. 228)
        \end{flushright}
    \section{Añade coma cuando sea necesario (razona tu propuesta):}
        Si te sientes irritable\color{verde}\textbf{,}\color{negro} \;no duermes bien\color{verde}\textbf{,}
        \color{negro}la tensión en el cuello y espalda se han vuelto insoportables e incluso tu sistema
        digestivo está en descontrol\color{verde}\textbf{,}\color{negro} \;seguramente estás sufriendo de
        ansiedad. Pero por suerte existen soluciones alternativas para prevenir o superar ese estado.

        (…)

        Si pruebas darte un masaje en el rostro o en las manos\color{verde}\textbf{,}\color{negro}
        \;descubrirás\color{verde}\textbf{,}\color{negro} \;con asombro\color{verde}\textbf{,}\color{negro}
        \;cuánta tensión se acumula en los lugares menos imaginados.
        \begin{flushright}
            (12-12-2011, “Tratamientos contra la ansiedad”, \textit{Aula Magna})
        \end{flushright}
        \clearpage

        En general se debe usar la coma para separar las distintas oraciones que se vayan formando y que
        tengan su propio sujeto y predicado. Por ejemplo, en \textit{(...)incluso tu sistema digestivo está en
        descontrol, seguramente estás sufriendo de ansiedad} la coma nos indica dónde comienza una nueva
        oración, con su sujeto (tú) y su predicado (seguramente estás sufriendo de ansiedad).

        Además de ello, la coma se utiliza para separar los elementos de una enumeración que van delante
        del último, que será separado por la conjunción \textit{y} seguramente. Por ejemplo, en \textit{Si te
        sientes irritable, no duermes bien, la tensión en el cuello y espalda se han vuelto insoportables e
        incluso tu sistema (...)}

        Finalmente hay otro uso de la coma más en este texto. Cuando queremos explicar algo o dar algún
        detalle más utilizamos un inciso determinado por sendas comas explicativas. Por ejemplo, en
        \textit{(...)descubrirás, con asombro, cuánta tensión se acumula (...)}

        \subsection*{¿Hay algún uso verbal en el texto anterior que te resulte extraño?}
            La mayor parte de los verbos están conjugados en presente, pero uso del pretérito perfecto en
            \textit{la tensión en el cuello y espalda se han vuelto insoportables} me ha llamado la atención
            por estar en una enumeración en la que los demás elementos tienen verbos en presente. Quizás
            sería más adecuado usar \textit{se vuelven}.

            También el uso de \textit{pruebas darte} en el segundo párrafo me parece incorrecto y lo
            sustituiría por \textit{pruebas a darte}.

    \section{Explica por qué Delibes ha usado las comas en los siguientes pasajes:}
        Por la Pascuilla, estuvo a punto de ocurrir en el pueblo una gran desgracia. Poco antes de comenzar la
        fiesta, el badajo de la campana golpeó la nuca del Antoliano y el Mamertito, el chico del Pruden, se
        deslizó desde la torre con el cable amarrado alrededor de la cintura.

        (…)

        A doña Resu, el Undécimo Mandamiento, le costó transigir con las imposiciones de Guadalupe, el
        Capataz, pero la decepción causada en los hombres del pueblo por el asunto del petróleo no se había
        disipado del todo (…).
        \begin{flushright}
            (Miguel Delibes, \textit{Las ratas}, Madrid, Destino, pg. 131)
        \end{flushright}

        Delibes utiliza las comas en alguno de los usos habituales como al cambiar el orden natural de la
        oración o hacer un inciso para explicar algo. Pero también hace uso de ella en lugar de los paréntesis
        en algunos lugares donde parecería más habitual utilizar los últimos. Por ejemplo, \textit{A doña
        Resu, el Undécimo Mandamiento, le costó transigir (...)}.

    \section{Puntúa el siguiente texto. Razona tus decisiones.}
        En sus inicios\color{verde}\textbf{,}\color{negro} \;la andadura de SEO/BirdLife Andalucía estuvo
        ligada a Doñana\color{verde}\textbf{,}\color{negro} \;hecho que no ha de sorprender a
        nadie\color{verde}\textbf{,}\color{negro} \;puesto que ya en el año 1954 los primeros ornitólogos
        españoles se reunieron ante el humedal más importante de Europa y constituyeron la primera asociación
        conservacionista en España\color{verde}\textbf{:}\color{negro} \;la Sociedad Española de
        Ornitología\color{verde}\textbf{,}\color{negro} \;hoy SEO/BirdLife.

        Esta Delegación dio sus primeros pasos en el mes de julio de 1998\color{verde}\textbf{,}\color{negro}
        \;tras uno de los episodios negros que ha sufrido nuestra tierra\color{verde}\textbf{:}\color{negro}
        \;la rotura de la balsa de inertes de las minas de Aznalcóllar. La Delegación de SEO/BirdLife en
        Andalucía cuenta con una oficina en Sevilla\color{verde}\textbf{,}\color{negro} \;una Oficina Técnica
        en el Centro Ornitológico Francisco Bernis\color{verde}\textbf{,}\color{negro} \;en el Espacio Natural
        Doñana\color{verde}\textbf{,}\color{negro} \;y una extensa red de grupos locales.

        \subsection*{Razonamiento}
            Hay varios usos de la coma en este texto. Voy a ir mencionando todos y dando ejemplos del propio
            texto. En primer lugar tenemos un cambio del orden natural de una oración por lo que se debe usar
            coma, \textit{En sus inicios, la andadura de SEO/BirdLife (...)}. También tenemos comas que deben
            a acompañar a locuciones causales como \textit{puesto que}. En el caso de \textit{(...) ligada a
            Doñana, hecho que no ha de sorprender a nadie\color{verde}\textbf{,}\color{negro} \;puesto que
            (...)} tenemos un inciso, aunque la última coma haga la doble función de delimitar el inciso y
            acompañar a la locución causal.

            Hay además otros usos como el de delimitar una matización en
            \textit{(...) la Sociedad Española de Ornitología, hoy SEO/BirdLife} o la habitual enumeración en
            \textit{(...) cuenta con una oficina en Sevilla, una Oficina Técnica en el Centro Ornitológico
            Francisco Bernis (...) y una extensa red (...)}. En esta oración también tenemos una coma con uso
            doble, ya que hay otro inciso que también está delimitado por comas: \textit{ (...) Centro
            Ornitológico Francisco Bernis, en el Espacio Natural Doñana\color{verde}\textbf{,}\color{negro}
            \;y una (...)}.

            También se usan los dos puntos en dos ocasiones. En ambos casos para la verificación o explicación
            de la oración anterior.
\end{document}

