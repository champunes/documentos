\documentclass[12pt, a4paper, oneside]{report}
\usepackage[spanish]{babel}
\usepackage[utf8x]{inputenc}
\usepackage[T1]{fontenc}
\usepackage{newtxtext}
\usepackage{etoolbox}
\usepackage[autostyle, spanish=spanish]{csquotes}
\usepackage{amsmath,amssymb,amsfonts}
\usepackage[margin=2.5cm]{geometry}
\usepackage{setspace}

\usepackage[table]{xcolor}

\definecolor{negro}{HTML}{000000} % #000000
\definecolor{rojo}{HTML}{ff0000} % #ff0000
\definecolor{verde}{HTML}{2da44e} % #2da44e


\title{Test de autoevaluación}
\author{José Ángel González Molina}
\date{\today}
\setlength{\parskip}{1em}
\setlength{\parindent}{1cm}
\onehalfspacing

\begin{document}
    \color{negro}
    \maketitle
    \clearpage

    \chapter*{Señala y corrige los errores de las siguientes oraciones}
    \setcounter{chapter}{1}

    \section{Nunca digas de \color{rojo}ese\color{negro} \;agua no beberé.}
        Agua es un sustantivo de género femenino, por lo que el determinante demostrativo que lo acompaña debe
        ser femenino también.
        \begin{center}
            \textit{Nunca digas de \color{verde}esa\color{negro} \;agua no beberé}
        \end{center}

        \section{Este chico tiene \color{rojo}mucho\color{negro} \;hambre.}
        Hambre es un sustantivo de género femenino, por lo que el determinante indefinido que lo acompaña debe
        ser femenino también.
        \begin{center}
            \textit{Este chico tiene \color{verde}mucha\color{negro} \;hambre}
        \end{center}

        \section{En el rastro se \color{rojo}vende\color{negro} \;monedas, revistas antiguas y toda
        clase de cosas.}
        Es una oración con pasiva refleja, por lo que el verbo debe mantener el número plural que tienen los
        sustantivos que forman parte del sujeto pasivo.
        \begin{center}
            \textit{En el rastro se \color{verde}venden\color{negro} \;monedas, revistas antiguas y
            toda clase de cosas}
        \end{center}

        \section{Las manzanas vienen en cajas de cartón \color{rojo}conteniendo\color{negro} \;cuatro piezas.}
        Los gerundios generalmente modifican el verbo. En este caso está modificando un sustantivo, cajas, por
        lo que se considera un gerundio adjetivo e incorrecto.
        \begin{center}
            \textit{Las manzanas vienen en cajas de cartón \color{verde}que contienen\color{negro}
            \;cuatro piezas}
        \end{center}

        \section{\color{rojo}La que\color{negro} \;no invitó fue a Marta, su mejor colaboradora.}
        La primera oración es una oración especificativa (indica que es Marta la persona a la que no invitó)
        por lo que es obligatorio usar una preposición.
        \begin{center}
            \textit{\color{verde}A la que\color{negro} \;no invitó fue a Marta, su mejor colaboradora}
        \end{center}

        \section{\color{rojo}Es\color{negro} \;por esa razón que los estudiantes se quejaron.}
        El tiempo verbal de la segunda oración es pretérito perfecto simple, por lo que lo correcto es que el
        tiempo verbal de la primera debería tener el mismo tiempo verbal.
        \begin{center}
            \textit{\color{verde}Fue\color{negro} \;por esa razón que los estudiantes se quejaron}
        \end{center}

        \section{Todos los campos eran \color{rojo}de nosotros\color{negro}.}
        En este caso lo correcto es hacer uso del pronombre posesivo adecuado.
        \begin{center}
            \textit{Todos los campos eran \color{verde}nuestros\color{negro}}
        \end{center}
        \clearpage

        \section{Estoy impaciente \color{rojo}de\color{negro} \;llegar a casa.}
        Después de impaciente se debe usar la preposición \emph{por}
        \begin{center}
            \textit{Estoy impaciente \color{verde}por\color{negro} \;llegar a casa}
        \end{center}

        \section{Han realizado todo en contra \color{rojo}tuyo\color{negro}.}
        No es correcta la combinación de en contra con posesivos masculinos, se debe usar el género femenino.
        \begin{center}
            \textit{Han realizado todo en contra \color{verde}tuya\color{negro}}
        \end{center}

        \section{He sopesado los pros y los \color{rojo}contra\color{negro} \;de mi decisión.}
        Pro y contra no deben usarse como invariables, por lo que en este caso deben respetar el número
        plural.
        \begin{center}
            \textit{He sopesado los pros y los \color{verde}contras\color{negro} \;de mi decisión}
        \end{center}
        \clearpage

        \setcounter{chapter}{2}
        \setcounter{section}{0}

        \section{Los diputados le preguntaron, \color{rojo}por fin, a sus colegas,\color{negro} \;si estaban
        de acuerdo con la medida.}
        El complemento indirecto habitualmente debe ir colocado detrás del verbo.
        \begin{center}
            \textit{Los diputados le preguntaron \color{verde}a sus colegas, por fin,\color{negro} \;si
            estaban de acuerdo con la medida}
        \end{center}

        \section{Lo que sucedió, \color{rojo}a grosso modo\color{negro}, fue que los asistentes se enojaron
        con el director y se marcharon.}
        Es incorrecto anteponer a la locución latina \emph{grosso modo} la preposición \emph{a}.
        \begin{center}
            \textit{Lo que sucedió, \color{verde}grosso modo\color{negro}, fue que los asistentes se
            enojaron con el director y se marcharon.}
        \end{center}

        \section{Me molesta el hecho de que nadie nos \color{rojo}da\color{negro} \;instrucciones.}
        Aquí es más correcto el uso de imperativo.
        \begin{center}
            \textit{Me molesta el hecho de que nadie nos \color{verde}dé\color{negro} \;instrucciones}
        \end{center}

        \section{Lo hizo todo \color{rojo}de motu propio\color{negro}.}
        Es incorrecto anteponer la preposición \emph{de} a la locución latina \emph{motu proprio}.
        \begin{center}
            \textit{Lo hizo \color{verde}motu proprio\color{negro}}
        \end{center}

        \section{No hay duda \color{rojo}que\color{negro} \;este trabajo es mejor que el anterior.}
        Tras la locución verbal \emph{haber duda}, es obligatorio el uso de la preposición \emph{de}.
        \begin{center}
            \textit{No hay duda \color{verde}de que\color{negro} \;este trabajo es mejor que el anterior}
        \end{center}

        \section{Ya era hora \color{rojo}que\color{negro} \;el gobierno interviniera.}
        Los complementos de nombre deben llevar la preposición \emph{de}.
        \begin{center}
            \textit{Ya era hora \color{verde}de que\color{negro} \;el gobierno interviniera}
        \end{center}

        \section{\color{rojo}Contra\color{negro} \;más se lee, menos se entiende.}
        No se debe usar la preposición \emph{contra} como sustituta de \emph{cuanto}.
        \begin{center}
            \textit{\color{verde}Cuanto\color{negro} \;más se lee, menos se entiende}
        \end{center}
        \clearpage

        \setcounter{chapter}{3}
        \setcounter{section}{0}

        \section{Estamos seguros \color{rojo}que\color{negro} \;la respuesta es correcta.}
        El complemento que expresa a qué se refiere estar seguro va precedido de la preposición \emph{de}.
        \begin{center}
            \textit{Estamos seguros \color{verde}de que\color{negro} \;la respuesta es correcta}
        \end{center}

        \section{\color{rojo}Bajo\color{negro} \;mi punto de vista, todos los hombres son iguales ante la
        ley.}
        Aunque el panhispánico de dudas acepta el uso de bajo para indicar un enfoque u opinión determinados,
        estilísticamente hay quien considera preferible desde según lo indicado en Fundéu.
        \begin{center}
            \textit{\color{verde}Desde\color{negro} \;mi punto de vista, todos los hombres son iguales ante la
            ley}
        \end{center}

        \section{Si quieres obtener el empleo, \color{rojo}debes de\color{negro} \;enviar dos currículums.}
        Para indicar obligación la forma correcta es \emph{deber + infinitivo}.
        \begin{center}
            \textit{Si quieres obtener el empleo, \color{verde}debes\color{negro} \;enviar dos currículums}
        \end{center}

        \section{\color{rojo}El alumno\color{negro} \;de quien te hablé, \color{rojo}lo\color{negro} \; han
        aprobado.}
        Como el sustantivo alumno ejerce de complemento indirecto del verbo aprobar, debe ir precedido de la
        preposición \emph{a}. Además el pronombre para referirse a él debe ser \emph{le}.
        \begin{center}
            \textit{\color{verde}Al alumno\color{negro} \;de quien te hablé, \color{verde}le\color{negro} \;han aprobado}
        \end{center}

        \section{Los recién llegados se pusieron detrás \color{rojo}nuestro\color{negro}.}
        Detrás es un adverbio por lo que es incorrecto el uso de un adjetivo posesivo para modificarlo. La
        forma correcta es usar el pronombre personal.
        \begin{center}
            \textit{Los recién llegados se pusieron detrás \color{verde}de nosotros\color{negro}}
        \end{center}

        \section{No me gusta mucho debatir sobre política.}

        \section{Por falta de espacio, he tenido que poner los libros \color{rojo}arriba\color{negro} \;del
        armario de la ropa.}
        Arriba se suele utilizar con preposiciones de movimiento y sin ningún otro sustantivo que indique una
        referencia. En este caso, como utilizamos el armario como referencia, el adverbio correcto es
        \emph{encima}. Otra expresión correcta sería \emph{en la parte de arriba} aunque por el contexto de la
        oración, ambas expresiones probablemente indiquen posiciones distintas.
        \begin{center}
            \textit{Por falta de espacio, he tenido que poner los libros \color{verde}encima\color{negro}
            \;del armario de la ropa}
        \end{center}
        \clearpage

        \setcounter{chapter}{4}
        \setcounter{section}{0}

        \section{Me olvidé \color{rojo}que\color{negro} \;debía indicarte la dirección de mi casa.}
        Cuando olvidar se usa como intransitivo pronominal (para expresar lo olvidado) debe ir precedido de la
        preposición \emph{de}.
        \begin{center}
            \textit{Me olvidé \color{verde}de que\color{negro} \;debía indicarte la dirección de mi casa}
        \end{center}

        \section{Estuvimos \color{rojo}adentro\color{negro} \;un buen rato, pero nos descubrieron.}
        Adentro indica hacia la parte interior por lo que se suele utilizar con verbos de movimiento. Para
        indicar en la parte interior lo correcto es usar \emph{dentro}.
        \begin{center}
            \textit{Estuvimos \color{verde}dentro\color{negro} \;un buen rato, pero nos descubrieron}
        \end{center}

        \section{No sé qué hora es. \color{rojo}Deben\color{negro} \;ser las doce.}
        Aunque actualmente está aceptado usar \emph{deber + infinitivo} para expresar una posibilidad, lo más
        correcto es añadir la preposición \emph{de}.
        \begin{center}
            \textit{No sé qué hora es. \color{verde}Deben de\color{negro} \;ser las doce}
        \end{center}

        \section{Han \color{rojo}cesado\color{negro} \;al último diputado provincial.}
        No se debe utilizar el verbo cesar como transitivo, por lo que lo correcto es usar alternativas como
        el verbo destituir.
        \begin{center}
            \textit{Han \color{verde}destituido\color{negro} \;al último diputado provincial}
        \end{center}
        \clearpage

        \section{La policía \color{rojo}explotó\color{negro} \;la bomba.}
        El verbo \emph{explotar} no se debe usar como transitivo, por lo que lo adecuado son las alternativas
        como \emph{explosionar} o \emph{hacer explotar}.
        \begin{center}
            \textit{La policía \color{verde}hizo explotar\color{negro} \;la bomba}
        \end{center}

        \section{La policía \color{rojo}incautó\color{negro} \;el alijo de drogas.}
        Aunque actualmente se acepta el uso de incautar como transitivo, la forma culta de uso es como
        intransitivo nominal y con un complemento de régimen introducido con la preposición \emph{de}.
        \begin{center}
            \textit{La policía \color{verde}se incautó del\color{negro} \;alijo de drogas}
        \end{center}

        \section{Es una persona \color{rojo}asequible\color{negro}.}
        No se debe usar asequible referido a una persona afable o de buen trato. Semánticamente la palabra
        correcta es \emph{accesible}.
        \begin{center}
            \textit{Es una persona \color{verde}accesible\color{negro}}
        \end{center}
\end{document}
